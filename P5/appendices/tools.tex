

\chapter{List of used tools}
\label{ch:tools}

\section{Shapely}
\label{tools:shapely}
Shapely \citep{Shapely2020} is a Python library used for the analysis and manipulation of planar features.
I use it to perform buffer operations in the client.

\section{cjio}
\label{tools:cjio}
Standing for CityJSON/io, \citep{cjio} is a Python\ac{cli} for processing and manipulating CityJSON datasets.

\section{Flask}
\label{tools:flask}
Flask \citep{Flask2020} is a lightweight Python web framework with which web applications can be built. 
It allows harp.gl to work together with cjio, as Flask acts as a server and enables the preparation of web pages.
%socketio

\section{three.js}
\label{tools:threejs}
A JavaScript library for the visualisation of 3D graphics in the browser, utilising the lower level \ac{webgl} \citep{Three.js2020}.
It is used in this thesis to visualise CityJSON datasets.

\section{harp.gl}
\label{tools:harpgl}
An open source 3D web map rendering engine for JavaScript \citep{harpgl}.
It utilises \ac{webgl} and three.js.
Used for the visualisation of CityJSON datsets.

\section{Turf.js}
\label{tools:turfjs}
It is a library written in JavaScript for geospatial analysis \citep{Turf.js2020}.
I have used it to buffer geometries in the client.

\section{Proj4js}
\label{tools:proj4js}
Proj4js \citep{Proj4js2020} is a JavaScript library that enables the transformation of coordinates to other \ac{crs}s.

\section{Selenium}
\label{tools:selenium}
Selenium \citep{Muthukadan2020} can drive web browsers in an automated way.
It has Python bindings and I have used it for the benchmarking part of the thesis.

\section{Draco}
\label{tools:draco}
An open-source library by Google \citep{dracoperformance} that is used to compess and decompress 3D geometric meshes for the purpose of improved storage and transmission efficiency.
It has functionalities in C++, JavaScript, and WebAssembly.
Used for the compression of geometries of CityJSON datasets.